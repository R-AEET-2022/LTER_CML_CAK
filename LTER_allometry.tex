% Options for packages loaded elsewhere
\PassOptionsToPackage{unicode}{hyperref}
\PassOptionsToPackage{hyphens}{url}
%
\documentclass[
]{article}
\usepackage{amsmath,amssymb}
\usepackage{lmodern}
\usepackage{iftex}
\ifPDFTeX
  \usepackage[T1]{fontenc}
  \usepackage[utf8]{inputenc}
  \usepackage{textcomp} % provide euro and other symbols
\else % if luatex or xetex
  \usepackage{unicode-math}
  \defaultfontfeatures{Scale=MatchLowercase}
  \defaultfontfeatures[\rmfamily]{Ligatures=TeX,Scale=1}
\fi
% Use upquote if available, for straight quotes in verbatim environments
\IfFileExists{upquote.sty}{\usepackage{upquote}}{}
\IfFileExists{microtype.sty}{% use microtype if available
  \usepackage[]{microtype}
  \UseMicrotypeSet[protrusion]{basicmath} % disable protrusion for tt fonts
}{}
\makeatletter
\@ifundefined{KOMAClassName}{% if non-KOMA class
  \IfFileExists{parskip.sty}{%
    \usepackage{parskip}
  }{% else
    \setlength{\parindent}{0pt}
    \setlength{\parskip}{6pt plus 2pt minus 1pt}}
}{% if KOMA class
  \KOMAoptions{parskip=half}}
\makeatother
\usepackage{xcolor}
\usepackage[margin=1in]{geometry}
\usepackage{color}
\usepackage{fancyvrb}
\newcommand{\VerbBar}{|}
\newcommand{\VERB}{\Verb[commandchars=\\\{\}]}
\DefineVerbatimEnvironment{Highlighting}{Verbatim}{commandchars=\\\{\}}
% Add ',fontsize=\small' for more characters per line
\usepackage{framed}
\definecolor{shadecolor}{RGB}{248,248,248}
\newenvironment{Shaded}{\begin{snugshade}}{\end{snugshade}}
\newcommand{\AlertTok}[1]{\textcolor[rgb]{0.94,0.16,0.16}{#1}}
\newcommand{\AnnotationTok}[1]{\textcolor[rgb]{0.56,0.35,0.01}{\textbf{\textit{#1}}}}
\newcommand{\AttributeTok}[1]{\textcolor[rgb]{0.77,0.63,0.00}{#1}}
\newcommand{\BaseNTok}[1]{\textcolor[rgb]{0.00,0.00,0.81}{#1}}
\newcommand{\BuiltInTok}[1]{#1}
\newcommand{\CharTok}[1]{\textcolor[rgb]{0.31,0.60,0.02}{#1}}
\newcommand{\CommentTok}[1]{\textcolor[rgb]{0.56,0.35,0.01}{\textit{#1}}}
\newcommand{\CommentVarTok}[1]{\textcolor[rgb]{0.56,0.35,0.01}{\textbf{\textit{#1}}}}
\newcommand{\ConstantTok}[1]{\textcolor[rgb]{0.00,0.00,0.00}{#1}}
\newcommand{\ControlFlowTok}[1]{\textcolor[rgb]{0.13,0.29,0.53}{\textbf{#1}}}
\newcommand{\DataTypeTok}[1]{\textcolor[rgb]{0.13,0.29,0.53}{#1}}
\newcommand{\DecValTok}[1]{\textcolor[rgb]{0.00,0.00,0.81}{#1}}
\newcommand{\DocumentationTok}[1]{\textcolor[rgb]{0.56,0.35,0.01}{\textbf{\textit{#1}}}}
\newcommand{\ErrorTok}[1]{\textcolor[rgb]{0.64,0.00,0.00}{\textbf{#1}}}
\newcommand{\ExtensionTok}[1]{#1}
\newcommand{\FloatTok}[1]{\textcolor[rgb]{0.00,0.00,0.81}{#1}}
\newcommand{\FunctionTok}[1]{\textcolor[rgb]{0.00,0.00,0.00}{#1}}
\newcommand{\ImportTok}[1]{#1}
\newcommand{\InformationTok}[1]{\textcolor[rgb]{0.56,0.35,0.01}{\textbf{\textit{#1}}}}
\newcommand{\KeywordTok}[1]{\textcolor[rgb]{0.13,0.29,0.53}{\textbf{#1}}}
\newcommand{\NormalTok}[1]{#1}
\newcommand{\OperatorTok}[1]{\textcolor[rgb]{0.81,0.36,0.00}{\textbf{#1}}}
\newcommand{\OtherTok}[1]{\textcolor[rgb]{0.56,0.35,0.01}{#1}}
\newcommand{\PreprocessorTok}[1]{\textcolor[rgb]{0.56,0.35,0.01}{\textit{#1}}}
\newcommand{\RegionMarkerTok}[1]{#1}
\newcommand{\SpecialCharTok}[1]{\textcolor[rgb]{0.00,0.00,0.00}{#1}}
\newcommand{\SpecialStringTok}[1]{\textcolor[rgb]{0.31,0.60,0.02}{#1}}
\newcommand{\StringTok}[1]{\textcolor[rgb]{0.31,0.60,0.02}{#1}}
\newcommand{\VariableTok}[1]{\textcolor[rgb]{0.00,0.00,0.00}{#1}}
\newcommand{\VerbatimStringTok}[1]{\textcolor[rgb]{0.31,0.60,0.02}{#1}}
\newcommand{\WarningTok}[1]{\textcolor[rgb]{0.56,0.35,0.01}{\textbf{\textit{#1}}}}
\usepackage{graphicx}
\makeatletter
\def\maxwidth{\ifdim\Gin@nat@width>\linewidth\linewidth\else\Gin@nat@width\fi}
\def\maxheight{\ifdim\Gin@nat@height>\textheight\textheight\else\Gin@nat@height\fi}
\makeatother
% Scale images if necessary, so that they will not overflow the page
% margins by default, and it is still possible to overwrite the defaults
% using explicit options in \includegraphics[width, height, ...]{}
\setkeys{Gin}{width=\maxwidth,height=\maxheight,keepaspectratio}
% Set default figure placement to htbp
\makeatletter
\def\fps@figure{htbp}
\makeatother
\setlength{\emergencystretch}{3em} % prevent overfull lines
\providecommand{\tightlist}{%
  \setlength{\itemsep}{0pt}\setlength{\parskip}{0pt}}
\setcounter{secnumdepth}{-\maxdimen} % remove section numbering
\ifLuaTeX
  \usepackage{selnolig}  % disable illegal ligatures
\fi
\IfFileExists{bookmark.sty}{\usepackage{bookmark}}{\usepackage{hyperref}}
\IfFileExists{xurl.sty}{\usepackage{xurl}}{} % add URL line breaks if available
\urlstyle{same} % disable monospaced font for URLs
\hypersetup{
  pdftitle={LTER\_allometry},
  pdfauthor={CML \& CAK},
  hidelinks,
  pdfcreator={LaTeX via pandoc}}

\title{LTER\_allometry}
\author{CML \& CAK}
\date{2022-09-22}

\begin{document}
\maketitle

\hypertarget{introduction}{%
\subsection{Introduction}\label{introduction}}

We selected the and\_vertebrates database from the LTER site (
\url{https://lter.github.io/lterdatasampler/reference/and_vertebrates.html}):

The dataset includes count and size data for cutthroat trout and
salamanders in clear cut or old growth sections of Mack Creek, Andrews
Forest LTER.

\begin{verbatim}
## here() starts at C:/Users/cris/Desktop/OptimizandoR_Sep22/LTER_CML_CAK
\end{verbatim}

\begin{verbatim}
## -- Attaching packages --------------------------------------- tidyverse 1.3.2 --
## v ggplot2 3.3.6      v purrr   0.3.4 
## v tibble  3.1.8      v dplyr   1.0.10
## v tidyr   1.2.1      v stringr 1.4.1 
## v readr   2.1.2      v forcats 0.5.2 
## -- Conflicts ------------------------------------------ tidyverse_conflicts() --
## x dplyr::filter() masks stats::filter()
## x dplyr::lag()    masks stats::lag()
## 
## Attaching package: 'plotly'
## 
## 
## The following object is masked from 'package:ggplot2':
## 
##     last_plot
## 
## 
## The following object is masked from 'package:stats':
## 
##     filter
## 
## 
## The following object is masked from 'package:graphics':
## 
##     layout
## 
## 
## 
## Attaching package: 'lubridate'
## 
## 
## The following objects are masked from 'package:base':
## 
##     date, intersect, setdiff, union
## 
## 
## # Attaching packages: easystats 0.5.2 (red = needs update)
## x insight     0.18.2    √ datawizard  0.6.0  
## x bayestestR  0.12.1    √ performance 0.9.2  
## √ parameters  0.18.2    √ effectsize  0.7.0.5
## √ modelbased  0.8.5     √ correlation 0.8.2  
## x see         0.7.2     √ report      0.5.5  
## 
## Restart the R-Session and update packages in red with `easystats::easystats_update()`.
## 
## 
## Loading required package: carData
## 
## lattice theme set by effectsTheme()
## See ?effectsTheme for details.
## 
## This is DHARMa 0.4.6. For overview type '?DHARMa'. For recent changes, type news(package = 'DHARMa')
## 
## New names:
## Rows: 32209 Columns: 17
## -- Column specification --------------------------------------------------------
## Delimiter: ","
## chr  (7): sitecode, section, reach, unittype, species, clip, notes
## dbl  (9): ...1, year, pass, unitnum, vert_index, pitnumber, length_1_mm, len...
## date (1): sampledate
## 
## i Use `spec()` to retrieve the full column specification for this data.
## i Specify the column types or set `show_col_types = FALSE` to quiet this message.
\end{verbatim}

\begin{verbatim}
## # A tibble: 6 x 17
##    ...1  year sitecode section reach  pass unitnum unittype vert_index pitnumber
##   <dbl> <dbl> <chr>    <chr>   <chr> <dbl>   <dbl> <chr>         <dbl>     <dbl>
## 1     1  1987 MACKCC-L CC      L         1       1 R                 1        NA
## 2     2  1987 MACKCC-L CC      L         1       1 R                 2        NA
## 3     3  1987 MACKCC-L CC      L         1       1 R                 3        NA
## 4     4  1987 MACKCC-L CC      L         1       1 R                 4        NA
## 5     5  1987 MACKCC-L CC      L         1       1 R                 5        NA
## 6     6  1987 MACKCC-L CC      L         1       1 R                 6        NA
## # ... with 7 more variables: species <chr>, length_1_mm <dbl>,
## #   length_2_mm <dbl>, weight_g <dbl>, clip <chr>, sampledate <date>,
## #   notes <chr>
\end{verbatim}

\hypertarget{plot-the-raw-data}{%
\subsection{Plot the raw data}\label{plot-the-raw-data}}

In this case, we are interested in modeling the length-mass
relationships for cutthroat trout and salamanders in Mack Creek:

\begin{verbatim}
## Warning: Removed 13279 rows containing missing values (geom_point).
\end{verbatim}

\includegraphics{LTER_allometry_files/figure-latex/plot raw data-1.pdf}

There are three species sampled but the Cascade torrent salamander is
almost absent from the dataset, so we decided to ignore this species
from our analysis.

First we subsampled the dataset and then plotted it:

\begin{Shaded}
\begin{Highlighting}[]
\NormalTok{data\_species2 }\OtherTok{\textless{}{-}}\NormalTok{ ourdata }\SpecialCharTok{\%\textgreater{}\%}
  \FunctionTok{subset}\NormalTok{(species }\SpecialCharTok{!=} \StringTok{"Cascade torrent salamander"}\NormalTok{)}

\NormalTok{data\_species2}\SpecialCharTok{$}\NormalTok{species }\OtherTok{\textless{}{-}} \FunctionTok{as.factor}\NormalTok{(data\_species2}\SpecialCharTok{$}\NormalTok{species)}
\FunctionTok{ggplot}\NormalTok{(data\_species2, }\FunctionTok{aes}\NormalTok{(}\AttributeTok{x =}\NormalTok{ length\_1\_mm,}
                    \AttributeTok{y =}\NormalTok{ weight\_g,}
                    \AttributeTok{color =}\NormalTok{ species)) }\SpecialCharTok{+}
  \FunctionTok{geom\_point}\NormalTok{()}
\end{Highlighting}
\end{Shaded}

\begin{verbatim}
## Warning: Removed 13270 rows containing missing values (geom_point).
\end{verbatim}

\includegraphics{LTER_allometry_files/figure-latex/subsample the dataset-1.pdf}

The relationship between length and weight of these species is
exponential, thus we decided to transform our variables into logarithm
(we checked log2 for length and log10 for biomass)

The reason for log10 transform the biomass is because we have many small
values (near zero) and wanted to enlarge the distribution ``scale''

\begin{Shaded}
\begin{Highlighting}[]
\NormalTok{data\_species\_log }\OtherTok{\textless{}{-}}\NormalTok{ data\_species2}
\NormalTok{data\_species\_log}\SpecialCharTok{$}\NormalTok{log\_length1 }\OtherTok{\textless{}{-}} \FunctionTok{log10}\NormalTok{(data\_species\_log}\SpecialCharTok{$}\NormalTok{length\_1\_mm)}
\NormalTok{data\_species\_log}\SpecialCharTok{$}\NormalTok{log\_weight }\OtherTok{\textless{}{-}} \FunctionTok{log10}\NormalTok{(data\_species\_log}\SpecialCharTok{$}\NormalTok{weight\_g)}

\FunctionTok{ggplot}\NormalTok{(data\_species\_log, }\FunctionTok{aes}\NormalTok{(}\AttributeTok{x =}\NormalTok{ log\_length1,}
                             \AttributeTok{y =}\NormalTok{ log\_weight,}
                             \AttributeTok{color =}\NormalTok{ species)) }\SpecialCharTok{+}
  \FunctionTok{geom\_point}\NormalTok{()}
\end{Highlighting}
\end{Shaded}

\begin{verbatim}
## Warning: Removed 13270 rows containing missing values (geom_point).
\end{verbatim}

\includegraphics{LTER_allometry_files/figure-latex/transform x,y data and plot it-1.pdf}

Result: It worked. The relationship is almost linear. We now model the
allometric relationship between the length and biomass of these species
and check whether the curves per species are different. We will check if
the models ``follow'' the assumptions :

\begin{Shaded}
\begin{Highlighting}[]
\NormalTok{lm\_log\_species }\OtherTok{\textless{}{-}} \FunctionTok{lm}\NormalTok{(log\_weight }\SpecialCharTok{\textasciitilde{}}\NormalTok{ log\_length1 }\SpecialCharTok{+}\NormalTok{ species, }\AttributeTok{data=}\NormalTok{data\_species\_log)}
\FunctionTok{summary}\NormalTok{(lm\_log\_species)}
\end{Highlighting}
\end{Shaded}

\begin{verbatim}
## 
## Call:
## lm(formula = log_weight ~ log_length1 + species, data = data_species_log)
## 
## Residuals:
##      Min       1Q   Median       3Q      Max 
## -1.77068 -0.03244  0.00066  0.03173  2.08377 
## 
## Coefficients:
##                         Estimate Std. Error t value Pr(>|t|)    
## (Intercept)            -4.299328   0.005009  -858.3   <2e-16 ***
## log_length1             2.922709   0.002860  1021.8   <2e-16 ***
## speciesCutthroat trout -0.554276   0.001184  -468.0   <2e-16 ***
## ---
## Signif. codes:  0 '***' 0.001 '**' 0.01 '*' 0.05 '.' 0.1 ' ' 1
## 
## Residual standard error: 0.07131 on 18918 degrees of freedom
##   (13270 observations deleted due to missingness)
## Multiple R-squared:  0.9823, Adjusted R-squared:  0.9823 
## F-statistic: 5.263e+05 on 2 and 18918 DF,  p-value: < 2.2e-16
\end{verbatim}

\begin{Shaded}
\begin{Highlighting}[]
\FunctionTok{check\_model}\NormalTok{(lm\_log\_species)}
\end{Highlighting}
\end{Shaded}

\includegraphics{LTER_allometry_files/figure-latex/first model-1.pdf}

Comentarlo (CARLOS)

\begin{Shaded}
\begin{Highlighting}[]
\NormalTok{lm\_log\_species\_i }\OtherTok{\textless{}{-}} \FunctionTok{lm}\NormalTok{(log\_weight }\SpecialCharTok{\textasciitilde{}}\NormalTok{ log\_length1 }\SpecialCharTok{*}\NormalTok{ species, }\AttributeTok{data=}\NormalTok{data\_species\_log)}
\FunctionTok{summary}\NormalTok{(lm\_log\_species\_i)}
\end{Highlighting}
\end{Shaded}

\begin{verbatim}
## 
## Call:
## lm(formula = log_weight ~ log_length1 * species, data = data_species_log)
## 
## Residuals:
##      Min       1Q   Median       3Q      Max 
## -1.77696 -0.03225  0.00075  0.03096  2.08983 
## 
## Coefficients:
##                                     Estimate Std. Error t value Pr(>|t|)    
## (Intercept)                        -4.157994   0.010196 -407.81   <2e-16 ***
## log_length1                         2.840679   0.005895  481.87   <2e-16 ***
## speciesCutthroat trout             -0.742230   0.011893  -62.41   <2e-16 ***
## log_length1:speciesCutthroat trout  0.106861   0.006728   15.88   <2e-16 ***
## ---
## Signif. codes:  0 '***' 0.001 '**' 0.01 '*' 0.05 '.' 0.1 ' ' 1
## 
## Residual standard error: 0.07084 on 18917 degrees of freedom
##   (13270 observations deleted due to missingness)
## Multiple R-squared:  0.9826, Adjusted R-squared:  0.9826 
## F-statistic: 3.556e+05 on 3 and 18917 DF,  p-value: < 2.2e-16
\end{verbatim}

\begin{Shaded}
\begin{Highlighting}[]
\FunctionTok{check\_model}\NormalTok{(lm\_log\_species\_i)}
\end{Highlighting}
\end{Shaded}

\includegraphics{LTER_allometry_files/figure-latex/unnamed-chunk-1-1.pdf}

COMENTARLO CARLOS

\end{document}
